%
% Niniejszy plik stanowi przykład formatowania pracy magisterskiej na
% Wydziale MIM UW.  Szkielet użytych poleceń można wykorzystywać do
% woli, np. formatujac wlasna prace.
%
% Zawartosc merytoryczna stanowi oryginalnosiagniecie
% naukowosciowe Marcina Wolinskiego.  Wszelkie prawa zastrzeżone.
%
% Copyright (c) 2001 by Marcin Woliński <M.Wolinski@gust.org.pl>
% Poprawki spowodowane zmianami przepisów - Marcin Szczuka, 1.10.2004
% Poprawki spowodowane zmianami przepisow i ujednolicenie 
% - Seweryn Karłowicz, 05.05.2006
% Dodanie wielu autorów i tłumaczenia na angielski - Kuba Pochrybniak, 29.11.2016

% dodaj opcję [licencjacka] dla pracy licencjackiej
% dodaj opcję [en] dla wersji angielskiej (mogą być obie: [licencjacka,en])
\documentclass[en]{pracamgr}

% Dane magistranta:
\autor{Wiktor Grzankowski}{429211}

\title{Stable Priceability for Additive Utilities}
\titlepl{Pojęcie Stable Priceability dla addytywnych użyteczności}

%\tytulang{An implementation of a difference blabalizer based on the theory of $\sigma$ -- $\rho$ phetors}

%kierunek: 
% - matematyka, informacyka, ...
% - Mathematics, Computer Science, ...
\kierunek{Computer Science}

% informatyka - nie okreslamy zakresu (opcja zakomentowana)
% matematyka - zakres moze pozostac nieokreslony,
% a jesli ma byc okreslony dla pracy mgr,
% to przyjmuje jedna z wartosci:
% {metod matematycznych w finansach}
% {metod matematycznych w ubezpieczeniach}
% {matematyki stosowanej}
% {nauczania matematyki}
% Dla pracy licencjackiej mamy natomiast
% mozliwosc wpisania takiej wartosci zakresu:
% {Jednoczesnych Studiow Ekonomiczno--Matematycznych}

% \zakres{Tu wpisac, jesli trzeba, jedna z opcji podanych wyzej}

% Praca wykonana pod kierunkiem:
% (podać tytuł/stopień imię i nazwisko opiekuna
% Instytut
% ew. Wydział ew. Uczelnia (jeżeli nie MIM UW))
\opiekun{dr hab. Piotr Skowron\\
  Faculty of Mathematics,\\
Informatics and Mechanics\\
  }

% miesiąc i~rok:
\date{September 2025}

%Podać dziedzinę wg klasyfikacji Socrates-Erasmus:
\dziedzina{ 
11.3 Informatyka
}

%Klasyfikacja tematyczna wedlug AMS (matematyka) lub ACM (informatyka)
\klasyfikacja{
\begin{itemize}
    \item Theory of computation
\end{itemize}
}

% Słowa kluczowe:
\keywords{elections, social choice theory, participatory budgeting, priceability, stable-priceability, method of equal shares}


% koniec definicji
\usepackage[utf8]{inputenc} 
\usepackage{polski}
\usepackage{graphicx}
\usepackage{amsmath}
\usepackage{amssymb}
\usepackage{amsfonts}
\usepackage{amsthm}
\usepackage[shortlabels]{enumitem}
\usepackage{mathdots}
\usepackage{url}
\usepackage{xfrac}
\usepackage{array}
\usepackage{hyperref} % referencje do miejsc w pdfie
\usepackage{environ} % zaawansowane środowiska
\usepackage{minted} % formatowanie kodu
\usepackage{tabularx} % lepsze tabelki
\usepackage{wrapfig} % do ustawiania pozycji obrazków
\usepackage{multicol} % wiele kolumn
\usepackage{csquotes}
\usepackage{booktabs}
\usepackage{colortbl}
\usepackage[table]{xcolor}
\usepackage{pifont}
\usepackage[
    backend=biber,
    style=numeric,
    sorting=none,
    urldate=long,
]{biblatex}

% --- Podstawowe matematyczne symbole
\newcommand{\wtw}{\quad \Leftrightarrow \quad}  % wtedy i tylko wtedy
\newcommand{\bigexists}{\mbox{\Large $\mathsurround0pt\exists$}\hspace{0.2em}} 
\newcommand{\bigforall}{\mbox{\Large $\mathsurround0pt\forall$}\hspace{0.1em}}
\newcommand{\lr}[1]{\left\langle #1 \right\rangle}
\renewcommand{\geq}{\geqslant}
\renewcommand{\leq}{\leqslant}

% --- Oznaczenia zbiorów
\newcommand{\RR}{\mathbb{R}}
\newcommand{\ZZ}{\mathbb{Z}}
\newcommand{\NN}{\mathbb{N}}
\newcommand{\CC}{\mathbb{C}}
\newcommand{\KK}{\mathbb{K}}
\newcommand{\QQ}{\mathbb{Q}}

% --- Tekst w math mode
\newcommand{\dm}[1]{\displaystyle{#1}}  % skrót do displaystyle
\newcommand{\textq}[1]{\quad \text{#1} \quad}
\newcommand{\textqq}[1]{\qquad \text{#1} \qquad}

% --- Kolory
\newcommand{\purple}[1]{\textcolor{purple}{#1}}
\newcommand{\teal}[1]{\textcolor{teal}{#1}}
\newcommand{\gray}[1]{\textcolor{gray}{#1}}
\newcommand{\grey}[1]{\textcolor{gray}{#1}}
\newcommand{\red}[1]{\textcolor{red}{#1}}
\newcommand{\orange}[1]{\textcolor{orange}{#1}}

\newcommand{\app}{\ding{51}}  % Approval symbol macro

% --- Kod Pythona
\newminted[python]{python}{
    autogobble,
    mathescape,
    frame = leftline,
    framerule = 1pt,
    framesep = 8pt
}
\newmintinline{python}{}
\addbibresource{bibliography.bib} 

\hypersetup{
    colorlinks=true,
    linkcolor={black},
    citecolor={blue},
    urlcolor={blue}
}

\newtheorem{definition}{Definition}[section]
\newtheorem{example}{Example}
\newtheorem{counterexample}{Counterexample}
\newtheorem{prooff}{Proof}
\newtheorem{theorem}{Theorem}
