\chapter*{Introduction}
\addcontentsline{toc}{chapter}{Introduction}
% Do rozwiniecia, wszystko 3-4 strony
Democratic decision‐making plays a central role in shaping modern societies. As participatory mechanisms such as participatory budgeting (PB) see increasing adoption across municipalities and regions worldwide~\cite{RussonGilman2012}, PB is widely recognized as a meaningful contribution to participatory democracy~\cite{PBCannabes}. \emph{Participatory budgeting}, invented in Porto Alegre in 1989~\cite{ParticipatoryBudgeting}, is a democratic process in which voters --- typically residents of a community --- select projects on which public funds should be spent.  Ensuring that these collective decisions fairly represent voters’ preferences has therefore become an important concern for self‐governing communities. These concerns have led both practitioners and researchers to seek formal mechanisms that can quantify and guarantee fairness in public‐spending decisions.

This gave rise to increased attention in \emph{computational social choice theory} --- an interdisciplinary field which uses methods that originate in computer science to examine collective decisions~\cite{AShortIntroductionToSocialChoice}. It focuses on formalizing various properties such as fairness and stability, by defining mathematical tools used to study elections and other collective choices.  One important direction in social choice theory is the formulation of desirable properties of election rules or allocations, called axioms~\cite{Brandt2016}. In mathematics, an axiom is a basic proposition assumed as a starting point for deductive reasoning that serves as a self-evident constraint. In social choice theory axioms aim to evaluate outcomes of elections and election rules in terms of their ability to reflect voters' will in a satisfying manner. 

Participatory budgeting elections can offer different kinds of \emph{ballots} to be cast~\cite{PBElicitation}. The most basic ones are \emph{approval ballots}, where voters express their support for a project, or the lack of it, in a binary way~\cite{Talmon_Faliszewski_2019}. There are also \emph{ordinal ballots}, where voters sort projects by their support for them. However, in this thesis, we mostly concentrate on \emph{cardinal ballots}. In the cardinal setting, voters can assign any positive value to projects they support, allowing for better expressiveness of their actual preferences. In practice, ordinal ballots can also be translated to cardinal ballots, assigning some number of points to each position on the sorted list~\cite{NSW2019}, as well as approval ballots can be seen as a special case of cardinal ones.

Just as there are multiple ways to cast votes, there are even more ways to count votes. The methods of counting votes are referred to as \emph{election rules}, which take as an input an election instance and return an allocation --- a set of projects ---  that are selected by that rule. Often allocations are divided into exhaustive and non-exhaustive ones. An allocation is said to be exhaustive, if no candidate can be added to the allocation within the initial budget. The first election rule we consider in the thesis is the \emph{Utilitarian Greedy} method, where the algorithm works as follows: sort in descending order all projects by the ratio of total utility assigned to it by all voters divided by its cost. Projects are then selected by the ratio. If adding the selected candidate does not exceed the budget limit, it is chosen. If not, it is removed and the next candidate is checked. The method terminates when the entire budget was spent or there are no more projects to choose from~\cite{DataToolsAndPB}. Utilitarian Greedy aims to maximize total satisfaction for voters and always returns exhaustive allocations, which follows directly from its definition.

The second method we consider is the \emph{Method of Equal Shares (MES)}~\cite{EqualShares}: an election rule aiming to lead to fair outcomes, representing true voters' preferences. MES is a sequential rule that lets voters “buy” projects using equal personal budgets. The process starts with an empty outcome and the same budget assigned for every voter. For each project, its supporters are willing to contribute up to their remaining budget, and --— if voters cast cardinal ballots --— the amount they offer grows in proportion to how much they value the project. A project becomes affordable as soon as its supporters can fully cover its cost. When that happens, the most affordable project --- the one with the smallest uniform charge across its supporters, proportional to utilities assigned --- is selected, funded, and its supporters are charged accordingly by reducing their remaining budgets. The procedure stops when no remaining project can be fully paid by its supporters. MES never overspends and may leave some budget unused. It is designed to give cohesive groups proportional influence rather than to pick the project with the highest total support.


The idea of axioms seems sensible, however, the meaning of fairness or proportionality is still not clearly defined. There are different approaches to this concept --- there are axioms, which help prevent some pathological situations, but do not provide very proportional representations in all cases. For instance, a proportionality axiom might demand that a group representing $x\%$ of the electorate see roughly $x\%$ of their preferred projects funded. An axiom called the core~\cite{AzizEtAl} states that "no group of voters should be able to deviate from the community and collectively choose another outcome, which they all would prefer over the outcome selected by the entire community". The core, like many other axioms~\cite{AzizEtAlComplexity}, is coNP-hard to verify. That means, given an election and an allocation, unless P=NP it is impossible to verify if that allocation is in the core, using a polynomial-time algorithm. In spite of its intuitive-sounding definition and high computational complexity, we are still aware of election instances with outcomes being in the core, that are remarkably unfair.  Empirically, election rules are also verified against different election instances to see how well they fulfil their goal of providing forms of fairness in the outcomes. Some metrics include: how many voters had $0$ of their approved projects selected? How much money was, on average, spent on each voters' approved projects? Was the money spent proportionally on different kinds of projects, such as bike lanes or playgrounds, with respect to the support that they had gathered?

The pursuit of truly proportional axioms accounting for different versions of voting ballots continues. In the last few years, numerous new concepts have been proposed including highly-intuitive market-based ones, namely \emph{priceability}~\cite{ProportionalityAndLimits} and \emph{stable priceability}~\cite{MarketBased}. They both start by defining a price system in which all voters are assigned the same fraction of the total budget, and payment functions, which describe how much each voter pays for each of the candidates. Priceability requires that there exists a price system, satisfying a number of conditions, which provide some basic form of proportional representations. These conditions sound very natural to any market-based system. That is, they require that voters pay only for candidates they support, that voters do not spend more than their budget, that each elected candidate must be fully funded by its supporters, no funds go towards unelected candidates and that for no unelected candidate, its supporters have enough leftover money to buy it. Stable priceability, or SP for short, is an extension of priceability, which adds a more complex stability constraint to the list of conditions required by priceability. It is more difficult to satisfy, but also provides greater proportionality. However, stable priceability is well-defined only for elections with approval ballots, which means that there are significant limitations to evaluating actual PB instances using the SP axiom, especially when multiple cities are allowing for casting cardinal ballots in their PB elections~\cite{StrasbourgBallots, ToulouseBallots, GdanskBallots}.

As computational evaluation of axioms is crucial for social choice theory, several programming frameworks have been developed for testing and comparison of voting rules and fairness properties in a convenient model. In particular, the open-source Pabutools Python library provides a comprehensive toolkit to work with PB instances~\cite{Pabutools}. That work is partially conducted using Integer Linear Programming (ILP) --- a mathematical optimization approach in which one maximizes or minimizes a linear objective function subject to linear constraints and integrality restrictions on the decision variables. ILP is especially powerful for social‐choice problems because it can encode complex budget constraints in a single model that high-performance solvers can tackle efficiently — even for NP‐hard instances. Importantly, Pabutools contains ILP encodings for several existing axioms. In this thesis, we use Pabutools for implementation purposes and the analysis of our work. 

The practical analysis often takes advantage of the \emph{cost utility} satisfaction measure, where the utility assigned to a project by a voter is multiplied by that projects' cost. Meaning that if a voter $v$ gave $2$ points to some project $c$, which costs $100$ units of currency, in computations we write that the actual satisfaction for $v$ from electing $c$ is equal to $200$. This satisfaction measure is based on the assumption, that in the PB setting, voters enjoy large projects more, as they affect them in a greater fashion~\cite{BrillForster2023}. However, we also consider the \emph{additive cardinal} satisfaction measure, where the satisfaction from projects is indifferent to their costs and depends only on the scores assigned to supported projects in the final allocation.

The primary goal of this thesis is to extend the market‐based fairness axiom of stable‐priceability from its original approval‐ballot setting to the more expressive domain of cardinal ballots, thereby enabling a richer modeling of voter utilities. To achieve this, we first formulate a strengthened ILP that incorporates cost‐weighted utility values and additional stability constraints. We then implement this extended definition in the open‐source Pabutools library. Finally, we carry out an extensive empirical study—using both non‐exhaustive and exhaustive budget scenarios—to measure how well the Utilitarian Greedy rule and several variants of the Method of Equal Shares satisfy stable priceability on real participatory budgeting datasets.

The remainder of this document is structured as follows:
\begin{enumerate}
  \item \textbf{Chapter~\ref{chap:1}} develops the formal election model and reviews the key social‐choice axioms that underpin our analysis.
  \item \textbf{Chapter~\ref{chap:2}} introduces the new stable priceability axiom for cardinal utilities, proves its fundamental properties, and contrasts it with the classical approval‐based definition.
  \item \textbf{Chapter~\ref{chap:3}} describes the software implementation in Pabutools and the comprehensive test suite used to validate the correctness of the code.
  \item \textbf{Chapter~\ref{chap:4}} presents our empirical evaluation: we apply the extended framework to a variety of real‐world PB instances, compare and analyze allocations returned by tested election rules.
  \item \textbf{Chapter~\ref{chap:5}} concludes with a summary of findings, a discussion of limitations, and an outline of promising directions for future research on relaxations and alternative voting rules.
\end{enumerate}
Throughout, we emphasize both formal proofs and reproducible code, with all experimental data and scripts made publicly available, attached in a repository at the end of the thesis.